\chapter{Source Code Twitter Streamer}
\label{app:B}

%selalu gunakan single spacing untuk source code !!!!!
\singlespacing 
% language: bahasa dari kode program
% terdapat beberapa pilihan : Java, C, C++, PHP, Matlab, R, dll
%
% basicstyle : ukuran font untuk kode program
% terdapat beberapa pilihan : tiny, scriptsize, footnotesize, dll
%
% caption : nama yang akan ditampilkan di dokumen akhir, lihat contoh
\begin{lstlisting}[language=Java,basicstyle=\tiny,caption=TwitterStreamerConsumer.java]

package com.wilianto.twitterstreamer;

import java.io.BufferedReader;
import java.io.FileWriter;
import java.io.IOException;
import java.io.InputStreamReader;
import org.scribe.builder.ServiceBuilder;
import org.scribe.builder.api.TwitterApi;
import org.scribe.model.OAuthRequest;
import org.scribe.model.Response;
import org.scribe.model.Token;
import org.scribe.model.Verb;
import org.scribe.oauth.OAuthService;

/**
 *
 * @author wilianto
 */
public class TwitterStreamConsumer extends Thread {

    private static final String STREAM_URL = "https://stream.twitter.com/1.1/statuses/filter.json";
    private static final String CONSUMER_KEY = "GpD9OBcSLvTR4O6K6fPIZkyuh";
    private static final String CONSUMER_SECRET = "hFNOtRXMUk8aSGGvyP7cIYOz8B2yarnqrpipBujvtAkXIgdgnt";
    private static final String ACCESS_TOKEN = "185124096-J1Ks92PJUfdn8ULlROFqRuUtZnwao6iVhpufsUXT";
    private static final String ACCESS_TOKEN_SECRET = "MeULu8Oc8l0LSq5fmB5V0CpsWYYBCWdQenJPczpEqpeAp";

    public void run() {
        try {
            System.out.println("Starting Twitter Streaming:");
            OAuthService service = new ServiceBuilder()
                    .provider(TwitterApi.class)
                    .apiKey(TwitterStreamConsumer.CONSUMER_KEY)
                    .apiSecret(TwitterStreamConsumer.CONSUMER_SECRET)
                    .build();

            //set access token
            Token accessToken = new Token(TwitterStreamConsumer.ACCESS_TOKEN, TwitterStreamConsumer.ACCESS_TOKEN_SECRET);

            //generate the request
            System.out.println("Connecting");
            OAuthRequest request = new OAuthRequest(Verb.POST, STREAM_URL);
            request.addHeader("version", "HTTP/1.1");
            request.addHeader("host", "stream.twitter.com");
            request.setConnectionKeepAlive(true);
            request.addHeader("user-agent", "Twitter Stream Reader");
            request.addBodyParameter("track", "libur,liburan,travel,travelling,wisata,tour,pariwisata,destinasi,hotel,pesawat");
            request.addBodyParameter("language", "in");
            service.signRequest(accessToken, request);
            Response response = request.send();

            BufferedReader reader = new BufferedReader(new InputStreamReader(response.getStream()));
            FileWriter fw = new FileWriter("part-"+System.currentTimeMillis()+".txt");
            String line;
            while ((line = reader.readLine()) != null) {
                System.out.println(line);
                fw.append(line);
            }
            fw.close();
        } catch (IOException ex) {
            ex.printStackTrace();
        }
    }
}

\end{lstlisting}

\begin{lstlisting}[language=Java,basicstyle=\tiny,caption=LetsStream.java]
package com.wilianto.twitterstreamer;

/**
 *
 * @author wilianto
 */
public class LetsStream {
    public static void main(String[] args) {
        TwitterStreamConsumer consumer = new TwitterStreamConsumer();
        consumer.start();
    }
}
\end{lstlisting}
