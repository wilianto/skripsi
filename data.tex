%_____________________________________________________________________________
%=============================================================================
% data.tex v7 (30-11-2015) \ldots dibuat oleh Lionov - Informatika FTIS UNPAR
%
% Perubahan pada versi 7 (30-11-2015)
%	- Perubahan nomor bagian karena penyisipan Bagian V
%	- Penambahan Bagian V : pengaturan kemunculan daftar gambar dan/atau tabel 
%	  dipindahkan dari main.tex
%	- Penambahan Bagian XV : tempat untuk menambahkan perintah yang dibuat 
%	  sendiri, dipindahkan dari main.tex
%	- Penambahan Bagian 0 : perintah bagi yang caption daftar isi tidak bisa
%	  ke bagian tengah
%
% Perubahan pada versi sebelumnya dapat dilihat di bagian akhir file ini
%_____________________________________________________________________________
%=============================================================================

%=============================================================================
% 								PETUNJUK
%=============================================================================
% Ini adalah file data (data.tex)
% Masukkan ke dalam file ini, data-data yang diperlukan oleh template ini
% Cara memasukkan data dijelaskan di setiap bagian
% Data yang WAJIB dan HARUS diisi dengan baik dan benar adalah SELURUHNYA !!
% Hilangkan tanda << dan >> jika anda menemukannya
%=============================================================================

%_____________________________________________________________________________
%=============================================================================
% 								BAGIAN 0
%=============================================================================
% PERHATIAN!! PERHATIAN!! Bagian ini hanya ada untuk sementara saja
% Jika "DAFTAR ISI" tidak bisa berada di bagian tengah halaman, isi dengan XXX
% jika sudah benar posisinya, biarkan kosong (i.e. \daftarIsiError{ })
%=============================================================================
\daftarIsiError{ }
%=============================================================================

%_____________________________________________________________________________
%=============================================================================
% 								BAGIAN I
%=============================================================================
% Tambahkan package2 lain yang anda butuhkan di sini
%=============================================================================
\usepackage{booktabs} 
\usepackage[table]{xcolor}
\usepackage{longtable}
\usepackage{amsmath}
\usepackage{todo}
\usepackage{multirow}
%=============================================================================

%_____________________________________________________________________________
%=============================================================================
% 								BAGIAN II
%=============================================================================
% Mode dokumen: menetukan halaman depan dari dokumen, apakah harus mengandung 
% prakata/pernyataan/abstrak dll (termasuk daftar gambar/tabel/isi) ?
% - kosong : tidak ada halaman depan sama sekali (untuk dokumen yang 
%            dipergunakan pada proses bimbingan)
% - cover : cover saja tanpa daftar isi, gambar dan tabel
% - sidang : cover, daftar isi, gambar, tabel (IT: UTS-UAS Seminar 
%			 dan UTS TA)
% - sidang_akhir : mode sidang + abstrak + abstract
% - final : seluruh halaman awal dokumen (untuk cetak final)
% Jika tidak ingin mencetak daftar tabel/gambar (misalkan karena tidak ada 
% isinya), edit manual di baris 439 dan 440 pada file main.tex
%=============================================================================
% \mode{kosong}
% \mode{cover}
% \mode{sidang}
%\mode{sidang_akhir}
\mode{final} 
%=============================================================================

%_____________________________________________________________________________
%=============================================================================
% 								BAGIAN III
%=============================================================================
% Line numbering: penomoran setiap baris, otomatis di-reset setiap berganti
% halaman
% - yes: setiap baris diberi nomor
% - no : baris tidak diberi nomor, otomatis untuk mode final
%=============================================================================
\linenumber{yes}
%=============================================================================

%_____________________________________________________________________________
%=============================================================================
% 								BAGIAN IV
%=============================================================================
% Linespacing: jarak antara baris 
% - single: opsi yang disediakan untuk bimbingan, jika pembimbing tidak
%            keberatan (untuk menghemat kertas)
% - onehalf: default dan wajib (dan otomatis) jika ingin mencetak dokumen
%            final/untuk sidang.
% - double : jarak yang lebih lebar lagi, jika pembimbing berniat memberi 
%            catatan yg banyak di antara baris (dianjurkan untuk bimbingan)
%=============================================================================
\linespacing{single}
%\linespacing{onehalf}
%\linespacing{double}
%=============================================================================

%_____________________________________________________________________________
%=============================================================================
% 								BAGIAN V
%=============================================================================
% Tidak semua skripsi memuat gambar dan/atau tabel. Untuk skripsi yang seperti
% itu, tidak diperlukan Daftar Gambar dan Daftar Tabel. Sayangnya hal ini 
% sulit dilakukan secara manual karena membutuhkan kedisiplinan pengguna 
% template.  
% Jika tidak akan menampilkan Daftar Gambar/Tabel, isi dengan NO. Jika ingin
% menampilkan, kosongkan parameter (i.e. \gambar{ }, \tabel{ })
%=============================================================================
\gambar{ }
\tabel{ }
%=============================================================================

%_____________________________________________________________________________
%=============================================================================
% 								BAGIAN VI
%=============================================================================
% Bab yang akan dicetak: isi dengan angka 1,2,3 s.d 9, sehingga bisa digunakan
% untuk mencetak hanya 1 atau beberapa bab saja
% Jika lebih dari 1 bab, pisahkan dengan ',', bab akan dicetak terurut sesuai 
% urutan bab (e.g. \bab{1,2,3}).
% Untuk mencetak seluruh bab, kosongkan parameter (i.e. \bab{ })  
% Catatan: Jika ingin menambahkan bab ke-10 dan seterusnya, harus dilakukan 
% secara manual
%=============================================================================
\bab{ }
%=============================================================================

%_____________________________________________________________________________
%=============================================================================
% 								BAGIAN VII
%=============================================================================
% Lampiran yang akan dicetak: isi dengan huruf A,B,C s.d I, sehingga bisa 
% digunakan untuk mencetak hanya 1 atau beberapa lampiran saja
% Jika lebih dari 1 lampiran, pisahkan dengan ',', lampiran akan dicetak 
% terurut sesuai urutan lampiran (e.g. \bab{A,B,C}).
% Jika tidak ingin mencetak lampiran apapun, isi dengan -1 (i.e. \lampiran{-1})
% Untuk mencetak seluruh mapiran, kosongkan parameter (i.e. \lampiran{ })  
% Catatan: Jika ingin menambahkan lampiran ke-J dan seterusnya, harus 
% dilakukan secara manual
%=============================================================================
\lampiran{ }
%=============================================================================

%_____________________________________________________________________________
%=============================================================================
% 								BAGIAN VIII
%=============================================================================
% Data diri dan skripsi/tugas akhir
% - namanpm: Nama dan NPM anda, penggunaan huruf besar untuk nama harus benar
%			 dan gunakan 10 digit npm UNPAR, PASTIKAN BAHWA BENAR !!!
%			 (e.g. \namanpm{Jane Doe}{1992710001}
% - judul : Dalam bahasa Indonesia, perhatikan penggunaan huruf besar, judul
%			tidak menggunakan huruf besar seluruhnya !!! 
% - tanggal : isi dengan {tangga}{bulan}{tahun} dalam angka numerik, jangan 
%			  menuliskan kata (e.g. AGUSTUS) dalam isian bulan
%			  Tanggal ini adalah tanggal dimana anda akan melaksanakan sidang 
%			  ujian akhir skripsi/tugas akhir
% - pembimbing: isi dengan pembimbing anda, lihat daftar dosen di file dosen.tex
%				jika pembimbing hanya 1, kosongkan parameter kedua 
%				(e.g. \pembimbing{\JND}{  } ) , \JND adalah kode dosen
% - penguji : isi dengan para penguji anda, lihat daftar dosen di file dosen.tex
%				(e.g. \penguji{\JHD}{\JCD} ) , \JND dan \JCD adalah kode dosen
% !!Lihat singkatan pembimbing dan penguji anda di file dosen.tex
%=============================================================================
\namanpm{Wilianto Indrawan}{2012730007}	%hilangkan tanda << & >>
\tanggal{6}{6}{2016}			%hilangkan tanda << & >>
\pembimbing{\VSM}{} %hilangkan tanda << & >>    
\penguji{\GDK}{\VAN} 				%hilangkan tanda << & >>
%=============================================================================

%_____________________________________________________________________________
%=============================================================================
% 								BAGIAN IX
%=============================================================================
% Judul dan title : judul bhs indonesia dan inggris
% - judulINA: judul dalam bahasa indonesia
% - judulENG: title in english
% PERHATIAN: - langsung mulai setelah '{' awal, jangan mulai menulis di baris 
%			   bawahnya
%			 - Gunakan \texorpdfstring{\\}{} untuk pindah ke baris baru
%			 - Judul TIDAK ditulis dengan menggunakan huruf besar seluruhnya !!
%			 - Gunakan perintah \texorpdfstring{\\}{} untuk baris baru
%=============================================================================
\judulINA{Sistem Kecerdasan Bisnis Pada Sistem Terdistribusi Hadoop, Studi Kasus: Teks Sosial Media Twitter \& Instagram}
\judulENG{Business Intelligence System on Hadoop Distributed System, Case Studies: Social Media Text Twitter \& Instagram}
%_____________________________________________________________________________
%=============================================================================
% 								BAGIAN X
%=============================================================================
% Abstrak dan abstract : abstrak bhs indonesia dan inggris
% - abstrakINA: abstrak bahasa indonesia
% - abstrakENG: abstract in english
% PERHATIAN: langsung mulai setelah '{' awal, jangan mulai menulis di baris 
%			 bawahnya
%=============================================================================
\abstrakINA{Sosial media saat ini sudah menjadi bagian dari kehidupan masyarakat. Hal-hal dalam kehidupan sehari-hari banyak dibagikan melalui sosial media, seperti foto makanan, foto atau posisi lokasi yang sedang dikunjungi, dan banyak hal lainnya yang dibagikan di sana. Saat ini sosial media dapat dianggap sebagai sumber informasi yang paling terkini dan paling jujur, karena ditulis langsung oleh penggunanya secara bebas. Oleh karena itu, informasi yang tersimpan di sosial media sangatlah berharga, sehingga perlu dimanfaatkan semaksimal mungkin untuk kepentingan organisasi. Salah satunya adalah mencari tren lokasi wisata yang sedang populer. Dua sosial media yang cukup banyak digunakan saat ini adalah Twitter dan Instagram.

Data mentah yang terdapat di dalam sosial media cukup besar dan perlu diproses dengan cepat, sehingga diperlukan sebuah sistem terdistribusi yang dapat mengolah data dari Twitter dan Instagram dengan cepat. Salah satu sistem terdistribusi yang populer saat ini adalah Apache Hadoop. Setelah data mentah diproses maka diperlukan sebuah sistem kecerdasan bisnis yang dapat digunakan untuk membuat laporan dan menganalisa trennya. 

Pada penelitian ini akan dibangun sistem kecerdasan bisnis pada Apache Hadoop dan diintegrasikan dengan perangkat lunak ekosistem Hadoop seperti Apache Hive dan Apache Sqoop, serta perangkat lunak Pentaho Data Integration. Sistem kecerdasan bisnis yang akan dibuat mencangkup proses awal pengambilan data sosial media yang relevan dari Twitter dan Instagram, pra pengolahan terhadap data tersebut, hingga menghasilkan laporan-laporan yang dapat dianalisis dalam bentuk tabel, grafik dan juga peta. Pada proses pengintegrasian data-data yang diambil dari kedua sosial media populer akan digunakan Pentaho Data Integration.
}
\abstrakENG{Now social media is a part of people. A lot of things in our daily life are shared at social media, such as food photos, destination photos or locations and many more. Social media can be considered the most up to date dan the most truthful information source, because it is written directly by the users independently. Because of that, the information at social media is very valuable, so it need to be used maximally for organization needs. One of them is for looking for trend of popular tourist location. Two social media which much used today are that Twitter and Instagram.

Raw data at social media is big enough and needs to be process quickly, so it need a distributed system which can process the data from Twitter and Instagram quickly. One of the popular distributed system today is Apache Hadoop. After the raw data has processed, it need a business intelligence system to make reports and analyzes the trend.

In this research, we will be built a business intelligence system on Apache Hadoop and will be integrated with applications, such as Apache Hive, Sqoop and Pentaho Data Integration. The business intelligence system covers relevant data collection from Twitter and Instagram, pre-process the data and generates reports which could be analyzed in the form of table, chart and map. Pentaho Data Integration will be used for integrating data from both social media.
} 
%=============================================================================

%_____________________________________________________________________________
%=============================================================================
% 								BAGIAN XI
%=============================================================================
% Kata-kata kunci dan keywords : diletakkan di bawah abstrak (ina dan eng)
% - kunciINA: kata-kata kunci dalam bahasa indonesia
% - kunciENG: keywords in english
%=============================================================================
\kunciINA{Kecerdasan Bisnis, Hadoop, Hive, Sqoop, Pentaho Data Integration, Twitter Streaming, Instagram Crawling}
\kunciENG{Business Intelligence, Hadoop, Hive, Sqoop, Pentaho Data Integration, Twitter Streaming, Instagram Crawling}
%=============================================================================

%_____________________________________________________________________________
%=============================================================================
% 								BAGIAN XII
%=============================================================================
% Persembahan : kepada siapa anda mempersembahkan skripsi ini ...
%=============================================================================
\untuk{dipersembahkan untuk masyarakat Indonesia}
%=============================================================================

%_____________________________________________________________________________
%=============================================================================
% 								BAGIAN XIII
%=============================================================================
% Kata Pengantar: tempat anda menuliskan kata pengantar dan ucapan terima 
% kasih kepada yang telah membantu anda bla bla bla ....  
%=============================================================================
\prakata{Puji syukur saya panjatkan kepada Tuhan Yang Maha Esa atas segala kasih dan karunia-Nya, sehingga saya dapat menyelesaikan skripsi ini tepat pada waktunya. Skripsi berjudul \textbf{Sistem Kecerdasan Bisnis pada Sistem Terdistribusi Hadoop, Studi Kasus: Teks Sosial Media Twitter \& Instagram} yang diajukan untuk memenuhi salah satu syarat untuk memperoleh gelar Sarjana Teknik pada jurusan Teknik Informatika di Fakultas Teknologi Informasi dan Sains (FTIS), Universitas Katolik Parahyangan. Pada kesempatan ini, saya mengucapkan terima kasih sebesar-besarnya kepada Yth: \begin{itemize}\item Kedua orang tua saya dan kedua adik saya yang telah senantiasa mendoakan dan mendukung saya selama proses pengerjaan skripsi ini.\item Ibu Dr. Veronica Sri Moertini selaku pembimbing utama yang telah meluangkan waktu untuk membantu, membimbing, dan memberikan saran dengan penuh kesabaran kepada saya selama pengerjaan skripsi ini.\item Bapak Gede Karya M.T.,CISA dan Ibu Vania Natali, M.T. selaku penguji saya yang telah bersedia meluangkan waktu untuk menguji dan memberikan masukan-masukan yang bermanfaat untuk skripsi ini.\item Dea Firda Rachmawati, selaku rekan dan sahabat yang telah menemani, mendukung dan membantu selama proses pengerjaan skripsi ini.\item Administrator laboratorium FTIS Unpar yang telah bersedia meluangkan waktu dan tenaganya membantu proses pengerjaan skripsi ini di laborarium FTIS.\item Sherly, Vinsensius Kevin, Adrian Aldo, dan Theodorus Kurniawan selaku teman seperjuangan yang telah saling membantu selama proses pengerjaan skripsi ini.\item Teman-teman kabinet dan staff Lembaga Kepresidenan Mahasiswa (LKM) periode 2015-2016 yang telah turut memberikan dukungan dalam proses pengerjaan skripsi ini.\item Liptia Venica, Herfan Heryandi, Fahrizal Septrianto, Adli Fariz, Florentina Sutanto, Hasan Putramas serta teman-teman mahasiswa seperjuangan lainnya yang telah mendukung, memberikan semangat dan membantu dalam banyak hal dalam proses penyusunan skripsi ini.\end{itemize}Saya berharap semoga skripsi ini dapat memberikan pengetahuan yang bermanfaat bagi para pembaca dan masyarakat luas. Saya selaku penulis menyadari bahwa penyusunan skripsi ini masih jauh dari sempurna. Oleh karena itu, saya mengharapkan kritik dan saran yang bersifat membangun sehingga dapat menjadi bekal bagi saya untuk masa mendatang.}
%=============================================================================

%_____________________________________________________________________________
%=============================================================================
% 								BAGIAN XIV
%=============================================================================
% Tambahkan hyphen (pemenggalan kata) yang anda butuhkan di sini 
%=============================================================================
\hyphenation{ma-te-ma-ti-ka}
\hyphenation{fi-si-ka}
\hyphenation{tek-nik}
\hyphenation{in-for-ma-ti-ka}
%=============================================================================

%_____________________________________________________________________________
%=============================================================================
% 								BAGIAN XV
%=============================================================================
% Tambahkan perintah yang anda buat sendiri di sini 
%=============================================================================
\newcommand{\vtemplateauthor}{lionov}
%=============================================================================

%=============================================================================
% Perubahan pada versi 6 (13-04-2015)
% 	- Perubahan untuk data-data ``template" menjadi lebih generik dan 
%	  menggunakan tanda << dan >>
% Perubahan pada versi 5 (10-11-2013)
% 	- Perbaikan pada memasukkan bab : tidak perlu menuliskan apapun untuk 
%	  memasukkan seluruh bab (bagian V)
% 	- Perbaikan pada memasukkan lampiran : tidak perlu menuliskan apapun untuk
%	  memasukkan seluruh lampiran atau -1 jika tidak memasukkan apapun
% Perubahan pada versi 4 (21-10-2012)
%	- Data dosen dipindah ke dosen.tex agar jika ada perubahan/update data 
%	  dosen, mahasiswa tidak perlu mengubah data.tex
%	- Perubahan pada keterangan dosen	
% Perubahan pada versi 3 (06-08-2012)
% 	- Perubahan pada beberapa keterangan 
% Perubahan pada versi 2 (09-07-2012):
% 	- Menambahkan data judul dalam bahasa inggris
% 	- Membuat bagian khusus untuk judul (bagian VIII)
% 	- Perbaikan pada gelar dosen
% Versi 1 (08-11-2011)
%=============================================================================