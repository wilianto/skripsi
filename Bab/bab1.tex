\chapter{Pendahuluan}
\label{chap:pendahuluan}

\section{Latar Belakang}
\label{sec:latar_belakang}
Pada saat ini, sosial media telah menjadi alat komunikasi yang sangat populer di kalangan pengguna internet. Setiap harinya jutaan orang saling berinteraksi satu sama lain melalui sosial media. Para pengguna sosial media terkadang menulis pesan tentang kehidupan pribadi, opini tentang berbagai topik dan pembahasan isu-isu yang sedang terjadi. Salah satu sosial media yang banyak digunakan oleh masyarakat adalah Twitter. Twitter adalah sosial media yang hanya memungkinkan pengguna menulis sebanyak 140 karakter saja, hal ini yang menyebabkan para pengguna sering menggunakan singkatan kata dan ejaan kata yang salah.

Pesan yang dibuat oleh para pengguna sosial media tentu saja dapat sangat bermanfaat bagi pemilik usaha, salah satunya di bidang pariwisata. Para pemilik usaha di bidang pariwisata dapat mengetahui tren wisata yang sedang terjadi di suatu wilayah. Selain itu mereka  juga dapat mempelajari pendapat masyarakat tentang suatu daerah wisata melalui pesan yang dibuat secara personal di sosial media. Pendapat negatif maupun positif yang didapatkan tentunya akan sangat bermanfaat bagi perusahaan untuk menunjang pengambilan keputusan. Namun pesan yang diperoleh dari sosial media jumlahnya sangat banyak dan berkembang dengan sangat cepat. Oleh karena itu pemrosesan pesan dari sosial media secara manual akan menyulitkan pemilik usaha untuk mendapatkan pengetahuan berharga yang mereka butuhkan untuk keperluan bisnis.

Pada skripsi ini akan dilakukan penelitian untuk membuat sistem yang dapat menganalisis pesan yang tidak terstruktur dari sosial media Twitter. Setelah itu sistem akan memasukan hasil analisisnya ke dalam sistem kecerdasan bisnis. Data terstruktur yang tersimpan di sistem kecerdasan bisnis akan memudahkan pemilik usaha untuk mempelajari hasil analisis pesan dan dapat melihat data dari suatu dimensi tertentu sesuai kriteria yang diperlukan. 

Sistem ini akan dibuat pada sistem terdistribusi Hadoop, karena besarnya jumlah dan ukuran pesan (\textit{big data}) yang diperoleh dari sosial media Twitter. Di dalam Hadoop terdapat dua bagian utama, yaitu MapReduce dan Hadoop Distributed File System (HDFS). MapReduce adalah model pemrograman rilisan Google yang ditujukan untuk memproses data berukuran raksasa secara terdistribusi dan paralel dalam \textit{cluster} yang terdiri atas ribuan komputer. HDFS adalah sebuah file sistem terdistribusi yang dirancang untuk berjalan di beberapa perangkat keras.

\section{Rumusan Masalah}
\label{sec:rumusan_masalah}
Berdasarkan latar belakang di atas, maka permasalahan yang muncul dapat dirumuskan ke dalam poin-poin berikut ini.
\begin{enumerate}
	\item Bagaimana pemanfaatan analisis data media sosial Twitter untuk menunjang pengambilan keputusan di bidang usaha pariwisata?
	\item Bagaimana mengembangkan teknik-teknik analisis \textit{big data} media sosial di dalam sistem terdistribusi Hadoop?
	\item Bagaimana membuat sistem kecerdasan bisnis yang dapat menunjang pengambilan keputusan di bidang usaha pariwisata?
	\item Bagaimana melakukan eksperimen untuk menguji performa sistem yang dibuat?
\end{enumerate}

\section{Tujuan}
\label{sec:tujuan}
Berdasarkan rumusan masalah di atas, maka tujuan dari topik skripsi ini adalah 
\begin{enumerate}
	\item membangun sistem untuk menganalisis pesan teks dari sosial media Twitter dengan memanfaatkan MapReduce di lingkungan sistem terdistribusi Hadoop
	\item membangun sistem kecerdasan bisnis yang dapat mengolah data hasil analisis menjadi data yang berguna dalam pengambilan keuputusan
\end{enumerate}

\section{Batasan Masalah}
\label{sec:batasan_masalah}
Agar penelitian skripsi ini tidak keluar dari pokok permasalahan yang dirumuskan, maka ruang lingkup pembahasan dibatasi pada:
\begin{enumerate}
	\item Teks yang dianalisis hanya teks yang berbahasa Indonesia saja. 
	\item Teks yang dianalisis diambil dari Twitter pada jangka waktu dan kata kunci tertentu.
\end{enumerate}

\section{Metodologi Penelitian}
\label{sec:metodologi_penelitian}
Dalam penelitian ini, dilakukan beberapa metode untuk memperoleh data atau informasi dalam menyelesaikan permasalahan yang muncul. Metode yang dilakukan tersebut antara lain:
\begin{enumerate}
	\item Studi pustaka yang dilakukan dengan cara mempelajari literatur-literatur baik buku, jurnal, artikel ilmiah, maupun artikel di website yang berkaitan dengan  \textit{text mining} dan Hadoop.
	\item Menganalisis pemanfaatan data sosial media yang berguna untuk menunjang pengambilan keputusan di bidang usaha pariwisata.
	\item Melakukan teknik-tenik pra pengolahan data.
	\item Merancang sistem untuk menganalisis data sosial media Twitter di dalam sistem terdistribusi Hadoop.
	\item Merancang sistem kecerdasan bisnis untuk menyimpan data hasil analisis.
	\item Melakukan eksperimen untuk menguji performa sistem yang dibuat.
	\item Menarik kesimpulan dan memberikan saran sesuai dengan penelitian yang telah dilakukan. 
\end{enumerate}

\section{Sistematika Pembahasan}
\label{sec:sistematika_pembahasan}
Sistematika penulisan skripsi ini dibagi menjadi x bab yaitu :\\
 \\
Bab 1 Pendahuluan\\ 
Bab1 berisi latar belakang, rumusan masalah, tujuan, batasan masalah, metodologi penelitian, dan sistematika pembahasan.
\\\\
Bab 2 Dasar Teori\\
Bab 2 berisi teori-teori yang akan mendukung pembahasan pada bab selanjutnya, seperti sosial media twitter, data mining, text mining, kecerdasan bisnis, sistem terdistribusi Hadoop dan Hive. 
\\\\
Bab 3 Analsis dan Eksperimen\\
Bab 3 berisi analisis domain masalah, analisis basis data, analisis review, analisis struktur data dan algoritma, dan pemodelan diagram use case.
\\\\
Bab 4 Perancangan\\
Bab 5 Implementasi dan Pengujian\\
Bab 6 Kesimpulan dan Saran berisi kesimpulan dari keseluruhan bab serta saran.\\
