\chapter{Analisis dan Rancangan Sistem Kecerdasan Bisnis}

\section{Analisis Masalah}
Keberadaan sosial media khususnya Twitter di Indonesia tidak dapat dipungkiri lagi sudah menjadi bagian hidup bagi kebanyakan orang. Masyarakat dapat menulis tentang apapun di sosial media miliknya, mulai dari opini pribadi tentang suatu barang atau jasa, membahas topik yang sedang hangat, ataupun hanya sekedar menuliskan lokasi keberadaannya saat ini. Namun sayangnya data yang tersebar di sosial media belum diolah dengan baik oleh para pemilik usaha. Padahal potensi yang dimiliki dari sosial media sangatlah besar bagi pertumbuhan usaha.

Salah satu sektor yang dapat memanfaatkan data dari sosial media adalah sektor pariwisata. Di sektor ini para pemilik usaha dapat memanfaatkan pesan dari sosial media untuk memprediksi tren wisata saat itu dan di masa depan. Selain itu mereka juga dapat mempelajari analisis sentimen negatif ataupun positif pengguna sosial media terhadap suatu lokasi wisata. 

Pesan media sosial saat ini khususnya Twitter jumlahnya sudah sangat banyak. Hal ini mengakibatkan pemilik usaha di bidang pariwisata kesulitan dalam mempelajari pesan-pesan tersebut secara manual. Masalah lainnya bagi mereka adalah pertumbungan pesan-pesan di sosial media yang berjalan dengan sangat cepat bahkan dalam hitungan detik datanya sudah berubah. 

Oleh karena itu, para pemilik usaha di bidang pariwisata memerlukan sebuah sistem yang dapat membantu mereka dalam mempelajari pesan-pesan di sosial media yang jumlahnya sangat banyak dan berkembang sangat cepat. Sistem yang dilengkapi dengan kecerdasan bisnis diharapkan dapat membantu para pemilik usaha di bidang pariwisata untuk melakukan pengambilan keputusan berdasarkan laporan, grafik dan hasil analisis sentimen dari pesan-pesan sosial media yang terkumpul di dalam \textit{datawarehouse}.

%\section{Analisis Data}

%\section{Analisis Sistem}