\chapter{Analisis Sistem Kecerdasan Bisnis}

\section{Analisis Data}
\subsection{Data Twitter}
Sumber data yang diambil dari Twitter menggunakan Twitter Streaming API akan menghasilkan data dalam format JavaScript Object Notation (JSON). Tabel \ref{tab:twitter_fields} adalah \textit{field-field} yang diambil untuk keperluan analisis pada penelitian ini. 

\begin{table}[!htb]
	\centering
	\begin{tabular}{| l | l | l |}
		\hline
		Field & Tipe & Deskripsi \\
	 	\hline
	 	id\_str & String & Unik \textit{identifier} untuk tweet dalam format string. \\
	 	text & String & Pesan twitter dengan encoding UTF-8. \\
	 	geo & Object & \textit{Nullable} Mendapatkan posisi pesan dibuat dalam bentuk\\
	 	& & latitude dan longitude. \\
	 	place & Places & \textit{Nullable}. Berisi data tempat yang berhubungan dengan\\
	 	& & tweet tapi tidak selalu asal lokasi tweet dibuat. \\
	 	timestamp\_ms & Timestamp & Berisi data waktu tweet dibuat dan ditampilkan dalam\\
	 	& & bentuk angka-angka (long).\\
	 	\hline
	\end{tabular}	
	\caption{Twitter Fields yang Digunakan}\label{tab:twitter_fields}
\end{table}	

\subsection{Data Instagram}
Sumber data lainnya yang digunakan untuk penelitian ini adalah data \textit{posting} dari Instagram yang biasanya disebut sebagai media. Instagram menyediakan banyak \textit{endpoint} yang dapat digunakan seperti users, relationships, media, comments, likes, tags dan locations. Untuk kasus pada penelitian ini penulis memilih \textit{endpoint} tags sebagai sumber data, karena data yang akan diambil adalah data dengan tags tertentu. Table \ref{tab:tag_endpoints} menjelaskan secara umum \textit{endpoint} apa saja yang terdapat di dalam kategori pencarian berdasarkan tag.

\begin{table}[H]
	\centering
	\begin{tabular}{| l | l | l |}
		\hline
		Method & URL & Deskripsi \\
		\hline
		GET & /tags/tag-name & Mendapatkan informasi tentang objek tag tersebut. \\
		GET & /tags/tag-name/media/recent & Mendapatkan sebuah \textit{list} media dengan tag tertentu. \\
		GET & /tags/search & Mencari variasi tag dengan keyword tertentu.\\
		\hline
	\end{tabular}	
	\caption{Tag Endpoints}\label{tab:tag_endpoints}
\end{table}	

Di antara tag \textit{endpoints} yang ada, yang paling relevan dengan kebutuhan analisis adalah tag \textit{endpoint} dengan kembalian \textit{list} media karena data dari medialah yang akan dianalisis. Untuk menggunakan \textit{endpoint} tersebut ada beberapa parameter yang harus diperhatikan (lihat Table \ref{tab:tag_endpoints_media}).

\begin{table}[H]
	\centering
	\begin{tabular}{| l | l |}
		\hline
		Parameter & Deskripsi \\
		\hline
		ACCESS\_TOKEN & Sebuah access token yang valid. \\
		COUNT & Jumlah media yang akan dikembalikan. \\
		MIN\_TAG\_ID & Mengembalikan media sebelum min\_tag\_id tertentu. \\
		MAX\_TAG\_ID & Mengembalikan media sesudah max\_tag\_id tertentu. \\
		\hline
	\end{tabular}	
	\caption{Tag Endpoint Media Recent}\label{tab:tag_endpoints_media}
\end{table}	

\textit{Endpoint} tersebut akan menghasilkan kembalian dalam format json berupa \textit{list} media-media. Setiap data media memiliki attribut-attribut yang dikembalikan seperti caption, image, dll (lihat Listing \ref{lst:response_tag_endpoint_media}). Pada penelitian ini data yang akan diteliti berdasarkan caption yang dibuat dan juga lokasi. Oleh karena itu data-data yang akan diambil pada penelitian ini hanyalah data pada atribut-atribut di tabel \ref{tab:instagram_fields}. 

Selain mengembalikan data-data media, \textit{endpoint} ini juga mengembalikan attribut \textit{pagination} yang berisi \textit{next\_url} dan data-data \textit{pagination} lainnya. Data tersebut digunakan untuk melakukan \textit{request} ke halaman berikutnya, informasi lebih detail tentang ini akan dibahas pada Bab \ref{sec:dfd_streaming_instagram}.

\begin{table}[H]
	\centering
	\begin{tabular}{| l | l | l |}
		\hline
		Field & Tipe & Deskripsi \\
	 	\hline
	 	id & String & Unik \textit{identifier} untuk media dalam format string. \\
	 	caption.text & String & Caption untuk media. \\
	 	location.name & String &  \textit{Nullable}. Berisi keterangan lokasi yang dimasukan oleh \\
	 	& & pengguna ketika mengunggah. \\
	 	created\_time & Timestamp & Berisi data waktu media dibuat dan ditampilkan dalam \\
	 	& & bentuk angka-angka (long).\\
		\hline
	\end{tabular}	
	\caption{Instagram Fields yang digunakan}\label{tab:instagram_fields}
\end{table}	

\begin{lstlisting}[basicstyle=\tiny,caption=Response Tag Endpoint Media Recent,label={lst:response_tag_endpoint_media}]
	{
	"pagination": {
		"next_url": "...",
		"prev_url": "...",
		...
	},
    "data": [{
        "type": "image",
        "users_in_photo": [],
        "filter": "Earlybird",
        "tags": ["snow"],
        "comments": {
            "count": 3
        },
        "caption": {
            "created_time": "1296703540",
            "text": "#Snow",
            "from": {
                "username": "emohatch",
                "id": "1242695"
            },
            "id": "26589964"
        },
        "likes": {
            "count": 1
        },
        "link": "http://instagr.am/p/BWl6P/",
        "user": {
            "username": "emohatch",
            "profile_picture": "http://distillery.s3.amazonaws.com/profiles/profile_1242695_75sq_1293915800.jpg",
            "id": "1242695",
            "full_name": "Dave"
        },
        "created_time": "1296703536",
        "images": {
            "low_resolution": {
                "url": "http://distillery.s3.amazonaws.com/media/2011/02/02/f9443f3443484c40b4792fa7c76214d5_6.jpg",
                "width": 306,
                "height": 306
            },
            "thumbnail": {
                "url": "http://distillery.s3.amazonaws.com/media/2011/02/02/f9443f3443484c40b4792fa7c76214d5_5.jpg",
                "width": 150,
                "height": 150
            },
            "standard_resolution": {
                "url": "http://distillery.s3.amazonaws.com/media/2011/02/02/f9443f3443484c40b4792fa7c76214d5_7.jpg",
                "width": 612,
                "height": 612
            }
        },
        "id": "22699663",
        "location": null
    },
    {
        "type": "video",
        "videos": {
            "low_resolution": {
                "url": "http://distilleryvesper9-13.ak.instagram.com/090d06dad9cd11e2aa0912313817975d_102.mp4",
                "width": 480,
                "height": 480
            },
            "standard_resolution": {
                "url": "http://distilleryvesper9-13.ak.instagram.com/090d06dad9cd11e2aa0912313817975d_101.mp4",
                "width": 640,
                "height": 640
            },
        "users_in_photo": null,
        "filter": "Vesper",
        "tags": ["snow"],
        "comments": {
            "count": 2
        },
        "caption": {
            "created_time": "1296703540",
            "text": "#Snow",
            "from": {
                "username": "emohatch",
                "id": "1242695"
            },
            "id": "26589964"
        },
        "likes": {
            "count": 1
        },
        "link": "http://instagr.am/p/D/",
        "user": {
            "username": "kevin",
            "full_name": "Kevin S",
            "profile_picture": "...",
            "id": "3"
        },
        "created_time": "1279340983",
        "images": {
            "low_resolution": {
                "url": "http://distilleryimage2.ak.instagram.com/11f75f1cd9cc11e2a0fd22000aa8039a_6.jpg",
                "width": 306,
                "height": 306
            },
            "thumbnail": {
                "url": "http://distilleryimage2.ak.instagram.com/11f75f1cd9cc11e2a0fd22000aa8039a_5.jpg",
                "width": 150,
                "height": 150
            },
            "standard_resolution": {
                "url": "http://distilleryimage2.ak.instagram.com/11f75f1cd9cc11e2a0fd22000aa8039a_7.jpg",
                "width": 612,
                "height": 612
            }
        },
        "id": "3",
        "location": null
    },
    ...]
}
\end{lstlisting}

\section{Analisis Masalah}
\label{sec:analisis_masalah}
Keberadaan sosial media khususnya Twitter dan Instagram di Indonesia sudah menjadi bagian hidup bagi kebanyakan orang. Masyarakat dapat menulis tentang apapun di sosial media miliknya. Namun sayangnya data yang tersebar di sosial media belum diolah dengan baik oleh para pemilik usaha. Padahal potensi yang dimiliki dari sosial media sangatlah besar bagi pertumbuhan usaha.

Salah satu sektor yang dapat memanfaatkan data dari sosial media adalah sektor pariwisata. Di sektor ini para pemilik usaha (\textit{travel agent}) dapat memanfaatkan pesan dari sosial media untuk mendapatkan informasi tren wisata saat itu. Selain para pemilik usaha, masih terdapat pihak-pihak lain yang dapat memanfaatkan pesan dari sosial media, seperti pemerintah melalui dinas pariwisata, wisatawan / \textit{backpacker} dan maskapai penerbangan atau penyedia layanan transportasi. 

Pesan media sosial saat ini khususnya Twitter dan Instagram jumlahnya sudah sangat banyak. Hal ini mengakibatkan pemilik usaha di bidang pariwisata kesulitan dalam menganalisis tren wisata yang sedang terjadi secara manual. Masalah lainnya bagi mereka adalah pertumbungan pesan-pesan di sosial media yang berjalan dengan sangat cepat bahkan dalam hitungan detik datanya sudah berubah. 

Oleh karena itu, para pemilik usaha di bidang pariwisata memerlukan sebuah sistem yang dapat membantu mereka dalam menganalisis tren wisata di sosial media yang jumlahnya sangat banyak dan berkembang sangat cepat. Sistem yang dilengkapi dengan kecerdasan bisnis diharapkan dapat membantu para pemilik usaha di bidang pariwisata untuk melakukan pengambilan keputusan berdasarkan laporan, grafik dan hasil analisis dari pesan-pesan sosial media yang terkumpul di dalam \textit{data warehouse}. Contoh pengetahuan-pengetahuan yang dapat dihasilkan dari sistem kecerdasan bisnis ini antara lain:

\begin{itemize}
	\item Mengetahui waktu-waktu ramai suatu lokasi wisata dikunjungi wisatawan. Dengan adanya informasi ini pemilik usaha dapat lebih mudah dalam melakukan promosi serta mempersiapkan pelayanan yang lebih baik untuk pelanggan ketika waktu-waktu ramai dan mengurangi pengeluaran-pengeluaran ketika waktu-waktu sepi. 
	\item Mengukur potensi wisata suatu daerah. Dengan adanya informasi ini pemerintah dapat melihat potensi-potensi wisata yang dimiliki suatu daerah, sehingga mempermudah dalam menentukan prioritas pembangunan di suatu lokasi wisata yang cukup favorit untuk dikunjungi.
	\item Mempersiapkan infrastruktur dan fasilitas yg memadai di saat-saat ramai. Dengan adanya informasi ini instansi pemerintah terkait juga dapat mempersiapkan dengen lebih matang infrastruktur, fasilitas, serta pengamanan di lokasi maupun akses menuju lokasi untuk kepuasan dan keselamatan para wisatawan selama berwisata. 
\end{itemize}
